\documentclass[11pt, a4paper]{article}
\usepackage{a4wide}
\usepackage{polski}
\usepackage[utf8]{inputenc}
\usepackage{amsthm}
\usepackage{amssymb}

\usepackage{semantic}

\newtheorem{zadanie}{Zadanie}

\author{Wojciech Jedynak \and Paweł Wieczorek}
\title{Ćwiczenia}

\begin{document}

\maketitle

\section*{Meta notatki}

Proponuję wybrać jakąś notację dla teorii typów Martina-L\"{o}fa. Może tę z książki ,,Programming in...''?


Czy chcemy mieć narysowane drzewa w proponowanych rozwiązań? W książce Martina-L\"{o}f-a on czasem rysuje drzewa
by coś pokazać. Musielibyśmy się zdecydować by było konsekwentnie.

Zadania do których nie zrobimy rozwiazań wzorcowych będą do wywalenia.

Zaznaczam przy zadaniach teoretycznych które można zrobić w Agdzie, możemy zrobić tak że dajemy ludziom te zadania
+ definicje w Agdzie, i ich wola czy zrobią tylko w Agdzie czy tylko na papierze, czy tu i tu.

\section{Agda i teoria}


\subsection{Proste}

\begin{zadanie}[Da się rozwiązać w Agdzie]
Ustalmy typy $A$ oraz rodzinę $B(a)$ dla każdego $a \in A$ oraz $C(a, a')$ dla każdych $a, a' \in A$.
Zbuduj termy o następujących typach:

\[
 \big( (\Pi x \in A)\;(\Pi y \in A)\; C(x, y) \big) \to (\Pi y \in A)\;(\Pi x \in A)\; C(x, y)\;
\]
\[
 \big( (\Sigma x \in A)\;(\Pi y \in A)\; C(x, y) \big) \to (\Pi y \in A)\;(\Sigma x \in A)\; C(x, y)\;
\]
\[
 \big( (\Pi x \in A)\;(\Pi y \in A)\;B(x) \to B(y)\big) \to (\Pi x \in A)\;(\Pi y \in A)\;\neg B(y) \to \neg B(x)
\]
\[
 \big( \neg (\Sigma x \in A)\;B(x)\big) \to (\Pi x \in A)\;\neg B(x)
\]
\[
 \big( (\Pi x \in A)\;B(x)\big) \to (\Pi x \in A)\;B(x)
\]
\end{zadanie}


\begin{zadanie}[Da się rozwiązać w Agdzie]
Ustalmy typy $A, B$, zbuduj termy o następujących typach:

\[
 (\Pi s \in A)\; s \equiv_A s
\]
\[
  (\Pi s \in A)(\Pi t \in A)\; s \equiv_A t \to t \equiv_A s
\]
\[
  (\Pi s \in A)(\Pi t \in A)(\Pi r \in A)\; s \equiv_A t \to t \equiv_A r \to s \equiv_A r
\]
\[
  (\Pi f \in A \to B)(\Pi s \in A)(\Pi t \in A)\; s \equiv_A t \to f\;t \equiv_B f\;s
\]

\end{zadanie}

\begin{zadanie}[Da się rozwiązać w Agdzie]
Zdefiniuj dodawanie i mnożenie na liczbach naturalnych, a następnie skonstruuj termy o następujących typach:

\[
 (\Pi m \in N)\;\mbox{add}\;0\;m \equiv m
\]
\[
 (\Pi n \in N)\;\mbox{add}\;n\;0 \equiv n
\]
\[
 (\Pi n \in N)\;(\Pi m \in N)\;\mbox{add}\;n\;m \equiv \mbox{add}\;m\;n
\]
\[
 (\Pi m \in N)\;\mbox{mult}\;0\;m \equiv 0
\]
\[
 (\Pi n \in N)\;\mbox{mult}\;n\;0 \equiv 0
\]
\[
 (\Pi n \in N)\;(\Pi m \in N)\;\mbox{mult}\;n\;m \equiv \mbox{mult}\;m\;n
\]
\[
 (\Pi n \in N)\;(\Pi m \in N)\;\mbox{mult}\;(\mbox{add}\;1\;n)\;m \equiv \mbox{add}\;n\;(\mbox{mult}\;m\;n)
\]
\[
 (\Pi n \in N)\;(\Pi m \in N)\;\mbox{mult}\;(\mbox{add}\;2\;n)\;m \equiv \mbox{add}\;(\mbox{mult}\; 2\; n)\;(\mbox{mult}\;m\;n)
\]

\end{zadanie}

\begin{zadanie}[Da się rozwiązać w Agdzie]
 Zdefiniuj poprzednik na liczbach naturalnych, a następnie zbuduj termy o następujących typach:

\[
  \mbox{pred}\;0 \equiv 0
\]
\[
  (\Pi n \in N)\;  \mbox{pred}\;(\mbox{add}\;1\;n) \equiv n
\]
\[
 (\Pi n \in N) 
\]

\end{zadanie}


\begin{zadanie}[Da się rozwiązać w Agdzie]
 Ustalmy typ $A$, zbubuj funkcję $toCh$ o następującym typie,
\[
  N \to (A \to A) \to A \to A
\]

a następnie termy o takich typach:
\[
 (\Pi f \in A \to A)\;(\Pi x \in A)\;\mbox{toCh}\;0\;f\;x \equiv x
\]
\[
 (\Pi f \in A \to A)\;(\Pi x \in A)\;\mbox{toCh}\;(\mbox{add}\;1\;n)\;f\;x \equiv f\;(\mbox{toCh}\;n\;f\;x)
\]
\[
 (\Pi n \in N)\;(\Pi m \in N)\;(\Pi f \in A \to A)\;(\Pi x \in A)\;
\mbox{toCh}\;(\mbox{add}\;n\;m)\;f\;x
\equiv
\mbox{toCh}\;n\;f\;(\mbox{toCh}\;m\;f\;x)
\]

\end{zadanie}


\begin{zadanie}[Da się rozwiązać w Agdzie]
 Ustalmy dowolne typy $A, B, C$.
 Pokaż, że typy $A \to B \to C$ oraz $A \times B \to C$ są izomorficzne. To jest, oprócz zdefiniowana dobrze
znanych funkcji zbuduj także dowód że są swoimi odwrotnościami, tzn termy o następujących typach:

\[
(\Pi f \in A \to B \to C)\; \mbox{curry}\; (\mbox{uncurry}\; f)\; \equiv f
\]

\[
(\Pi f \in A \times B \to C)\; \mbox{uncurry}\; (\mbox{curry}\; f)\; \equiv f
\]

%\[
% (\Pi f \in A \to B \to C)\;(\Pi x \in A)\;(\Pi y \in B)\; \mbox{curry}\; (\mbox{uncurry}\; f)\;x\;y \equiv f\;x\;y
%\]

%\[
% (\Pi f \in A \times B \to C)\;(\Pi x \in A)\;(\Pi y \in B)\; \mbox{uncurry}\; (\mbox{curry}\; f)\;(x,y) \equiv f\; (x,y)
%\]

(sprawdzić czy się da, czy trzeba dodać jeszcze argumenty, tzn $ (\cdots f \cdots )\;x\;y \equiv f \;x\;y$
\end{zadanie}

\subsection{Bardziej teoretyczne}

\begin{zadanie}[Da się rozwiązać w Agdzie]
 Ustalmy typ $A$, zakoduj za pomocą $W$-typów typ $Maybe\; A$ znany z Haskella.
\end{zadanie}


\begin{zadanie}[NIE da się rozwiązać w Agdzie]
 W książce ,,Intuinistic type theory'' pojawia się dodatkowa reguła wnioskowania dotyczaca równości:

\begin{center}
\begin{tabular}{c}
\inference{
H \in x \equiv_A y
}
{
 x = y : A
}
\end{tabular}
\end{center}

Mówi ona, że jeżeli posiadamy dowód, że dwa termy są równe to są one konwertowalne. Ta reguła sprawia, że
type-checking jest nierozstrzygalny, a taką równość i teorię typów nazywany ekstensjonalną. 
Korzystając z tej reguły
udowodnij ekstensjonalność funkcji, tzn pokaż że w ekstensjonalnej teorii typów dla ustalonych typów $A, B$ możemy
zbudować term o typie

\[
(\Pi f \in A \to B)\;(\Pi g \in A \to B)\;\big( (\Pi x \in A)\; f\; x \equiv g\; x \big) \to f \equiv g
\]

Zadania nie da się rozwiązać w~Agdzie ponieważ nie można rozszerzać systemu o nowe reguły wnioskowania.
(Sprawdzić czy $\eta$-ekspansja jest potrzebna, czy jest dowodliwa z tą równością)
\end{zadanie}

\begin{zadanie}
 Skonstruuj negację bitową, tzn term $\mbox{negb} : N_2 \to N_2$ a następnie skontruuj term o typie:
\[
 (\Pi b \in N_2) \neg (b \equiv \mbox{negb}\;b)
\]

Wskazówka: Trudność to dowód że $0_2$ jest różne od $1_2$, ponieważ nie mamy typów indukcyjnych to to nie jest
oczywiste, trzeba użyć uniwersum jak z czwartym aksjomatem Peano  na seminarium.
\end{zadanie}

\begin{zadanie}[Da się rozwiązać w Agdzie]
 Udowodnij, że nie istnieje funkcja z liczb naturalnych w ciągi zero-jedynkowych, która ma funkcję odwrotną.
To jest skonstruuj
 term $thm$ o następującym typie:
\[
 \neg (\Sigma f \in N \to BinSeq)\, (\Sigma g \in BinSeq \to N)\,
(\Pi s \in BinSeq)\,f (g\, s) \equiv s 
\]
gdzie $BinSeq$ oznacza $N \to N_2$.

Wskazówka: Dowód tego twierdzenia to standardowy przykład metody przekątniowej, można znaleźć rozwiązanie w Whitebooku.
Trudność polega na przeniesieniu tego na naturalną dedukcję (Zaskakujące może być, że to twierdzenie jest konstruktywne!).

\end{zadanie}

\begin{proof}[Propozycja rozwiązania]
Zanim sformalizujemy metodę przekątniową przypomnijmy sobie ten dowód, by ustalić co konkretnie chcemy uzyskać.

Pokażmy, że dla dowolnych funkcji $f$ i $g$ potrafimy dojśc do sprzeczności o ile $g$ jest odwrotnością $f$.
Stwórzmy ,,przekątniowy'' ciąg zero-jedynkowy $h : BinSeq$:
\[
 h = \lambda n. \mbox{negb}\;(f\;n\;n)
\]
Element tego ciągu o numerze $n$ jest równy negacji $n$-tego elementu w~$n$-tym ciągu w~wyznaczonej numeracji
przez funkcję $f$. Za pomocą funkcji $g$ możemy uzyskać numer tego ciągu, niech $n_h = g\;h$. Teraz, zauważmy że
\[
h\;n_h = \mbox{negb}\;(f\;n_h\;n_h) = \mbox{negb}\;(f\;(g\;h)\;n_h) = \mbox{negb}\;(h \;n_h)
\]

czyli sprzeczność.


Możemy teraz zacząć zastanawiać się jak przenieść powyższe rozumowanie na naturalną dedukcję, musimy
spróbować skonstruować
funkcję $diagonal$ o następującym typie:
{
\small
\[
 (\Pi f \in N \to BinSeq)\, (\Pi g \in BinSeq \to N)\,
\to
\big(
(\Pi s \in BinSeq)\,f (g\, s) \equiv s 
\big)
\to N_0
\]
}

Chcemy aby odpowiadała ona przedstawionemu rozumowaniu. Term zacząć pisać prosto:

\[
  \mbox{diagonal} = \lambda f.\lambda g. \lambda C.\; \framebox{?}
\]

W miejscu \framebox{?} chcemy skonstruować absurdalną wartość. Ale jak? Sprzeczność uzyskaliśmy
pokazując, że $h\;n_h = \mbox{negb}\; (h\;n_h)$, bo wiemy że dla dowolnego $b$ zachodzi $b \not= \mbox{negb}\;b$.

Wykorzystajmy te szczegóły w praktyce: z poprzedniego zadania mamy term
$\mbox{Hnegb} : (\Pi b \in N_2) \neg (b \equiv \mbox{negb}\;b)$, czyli pamiętając co rozumiemy
jako negację - jesteśmy w posiadaniu metody, która z dowodu $b \equiv \mbox{negb}\;b$ konstruuje
absurdalną wartość. Wykorzystajmy tę metodę dla $h\;n_h$.

\[
  \mbox{diagonal} = \lambda f.\lambda g. \lambda C.\; \mbox{Hnegb}\; (h\;n_h)\; Heq
\]
gdzie
\[
 Heq = \framebox{?} : h\;n_h \equiv \mbox{negb}\; (h\;n_h)
\]

By skonstruować świadka tej równości musimy przeanalizować ciąg równości w~oryginalnym rozumowaniu - pierwsze dwie
\[
h\;n_h = \mbox{negb}\;(f\;n_h\;n_h) = \mbox{negb}\;(f\;(g\;h)\;n_h)
\]
mamy z definicji $h$ oraz $n\_h$. Czyli interesuje nas jedynie term typu
\[
\mbox{negb}\;(f\;(g\;h)\;n_h) \equiv \mbox{negb}\; (h\;n_h)
\]

...

--

Dowód twierdzenia, to funkcja która ze świadka istnienia takich funkcji $f$ i $g$ ma konstruować wartość absurdalną, która
jedyne co musi zrobić to rozpakować dane z~argumentu i zaaplikować do nich funkcję $diagonal$. 

\[
thm = 
 \lambda H.\;
 \Sigma\mbox{-elim}\;
(\lambda f. \lambda H'.\; \Sigma\mbox{-elim}\; (\lambda g. \lambda H''.\;
\mbox{diagonal}\;f\;g\;H''
)\;
H')\;H
\]

\end{proof}

\section{Tylko Agda}


\end{document}


