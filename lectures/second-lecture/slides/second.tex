\documentclass{beamer}

\usetheme{Boadilla}

% brak tej dziwnej nawigacji w prawym dolnym rogu 
\setbeamertemplate{navigation symbols}{} 

% brak zbędnych informacji w stopce
\setbeamertemplate{footline}[page number]{}

% klikalne linki
\usepackage{hyperref}

% kodowanie utf8 
\usepackage[utf8]{inputenc} 
\usepackage{polski}
\usepackage{amsthm}
\usepackage{semantic}

\newtheorem{thm}{Twierdzenie}

% konfiguracja niektórych parametrów 
% \setbeamercovered{transparent} 
\setbeamertemplate{bibliography item}[text] 
\setlength{\parskip}{1ex} 
\setlength{\parindent}{0pt} 

\title{Seminarium: Programowanie w teorii typów}
\subtitle{Teoria typów}
\author{Wojciech Jedynak, Paweł Wieczorek} 
\institute{Instytut Informatyki Uniwersytetu Wrocławskiego}

%\AtBeginSection[]
%{
   %\begin{frame}
       %\frametitle{}
       %\tableofcontents[sectionstyle=show/hide, subsectionstyle=show/show/hide]
   %\end{frame}
%}

\begin{document}

%\lstset{language=Lisp}

\maketitle

%%%%%%%%%%%%%%%%%%%%%%%%

% \tableofcontents[hidesubsections]

%%%%%%%%%%%%%%%%%%%%%%%%%

\begin{frame}
\frametitle{Plan}

\begin{itemize}
 \item Przypomnimy sobie ideę konstruktywizmu oraz izomorfizmu Curry'ego-Howarda.
 \item Zapoznamy się z podstawami teorii typów Martina-L\"{o}fa:
 \begin{itemize}
 \item Zapoznamy się z sądami w tym systemie.
 \item Sformułujemy podstawowe typy jak liczby naturalne.
 \item Przebrniemy przez formalne definicje podstawowych typów zależnych: $\Pi$-typ, $\Sigma$-typ.
 \item Zobaczymy jak teoria typów koduje logikę pierwszego rzędu.
 \end{itemize}
\end{itemize}

\end{frame}

\section{Matematyka konstruktywna}

\begin{frame}
\frametitle{Matematyka konstruktywna}

\begin{itemize}
 \item powstały na początku poprzedniego wieku pogląd na temat fundamentów matematyki
 \item L.E.J.Brouwer, twórca ideologii
 \item empiryczna zawartość twierdzeń matematycznych
 \item co znaczy orzeczenie o istnieniu pewnego obiektu?
 \item odrzucenie dowodów przez sprowadzenie do sprzeczności
 \item odrzucenie idealistycznego podejścia do prawdziwości orzeczeń
 \item E. Bishop, konstruktywna analiza matematyczna
\end{itemize}


\end{frame}

%%%%%%%%%%%%%%%%%%%%%%%%%

\begin{frame}
\frametitle{Książkowy przykład twierdzenia niekonstruktywnego}

\begin{thm}
 Istnieją takie dwie liczby niewymierne $a$ oraz $b$, że $a^b$ jest liczbą wymierną.
\end{thm}

\begin{proof}
 Orzeczenie, że $\sqrt{2}^{\sqrt{2}} \in \mathbf{Q}$ musi być prawdziwe lub musi być fałszywe.
\begin{itemize}
 \item jeżeli jest prawdziwe to mamy szukane $a$ oraz $b$
 \item jeżeli jest fałszywe to niech $a = \sqrt{2}^{\sqrt{2}}$ oraz $b = \sqrt{2}$, wtedy $a^b = 2$
\end{itemize}

\end{proof}

Jedyne co wiemy, to to że muszą istnieć takie liczby. 

\end{frame}

%%%%%%%%%%%%%%%%%%%%%%%%%

\begin{frame}
\frametitle{Kolejny przykład.}

\begin{thm}[klasycznie]
 Jeżeli funkcja $f$ jest ciągła na przedziale $[0, 1]$ oraz wartości funkcji na~krańcach przedziału mają różne znaki
to istnieje punkt w~tym przedziale na~którym funkcja się zeruje.
\end{thm}

\begin{thm}[konstruktywnie]
 Jeżeli funkcja $f$ jest ciągła na przedziale $[0, 1]$ oraz wartości funkcji na~krańcach przedziału mają różne znaki
to dla każdego $\epsilon > 0$ istnieje punkt w~tym przedziale na~którym bezwzględna wartość funkcji jest mniejsza od
$\epsilon$.
\end{thm}

\end{frame}

%%%%%%%%%%%%%%%%%%%%%%%%%

\begin{frame}
\frametitle{Interpretacja Brouwer-Heyting-Kołmogorow}

\begin{itemize}
 \item $A \wedge B$ to konstrukcja składająca się z~dwóch pod-konstrukcji
 \item $A \vee B$ to konstrukcja składająca się z lewej lub prawej pod-konstrukcji
 \item $A \to B$ to metoda przekształcająca konstrukcję $B$ mając do~dyspozycji $A$
 \item $\bot$ absurd, konstrukcja której nie można zrealizować
 \item $\forall x. P(x)$ to metoda przekształcająca wartość $a$ w~konstrukcję $P(a)$
 \item $\exists x. P(x)$ to konstrukcja mająca składać się ze świadka $a$ oraz z konstrukcji $P(a)$
 \item $\neg A$ to skrót od $A \to \bot$ 
 \item Czy przy tej interpretacji wszystkie klasyczne prawa mają sens?
\begin{itemize}
\item $\exists x\;P(x) \vee \neg \exists x\;P(x)$
\item $\exists x\;P(x) \equiv \neg (\forall x \neg P(x))$
\item $A \vee B \equiv \neg (\neg A \wedge \neg B)$
\end{itemize}

\end{itemize}


\end{frame}

%%%%%%%%%%%%%%%%%%%%%%%%%

\begin{frame}
\frametitle{System naturalnej dedukcji}

\begin{itemize}
 \item system dowodzenia
 \item posługujemy się sądami $\Gamma |- \varphi$
 \item dowód to wyprowadzenie o~strukturze drzewa
 \begin{itemize}
 \item korzeń - wniosek (sąd)
 \item węzeł - reguła wnioskowania
 \item liść - aksjomat
 \end{itemize}
 \item reguły wprowadzania i eliminacji spójników logicznych
\end{itemize}

\begin{center}
\begin{tabular}{lr}
\inference{
\inference{
\dfrac{A \wedge B |- A \wedge B}{A \wedge B |- B} \qquad \dfrac{A \wedge B |- A \wedge B}{A \wedge B |- A}
}
{
A \wedge B |- B \wedge A
}
}
{
|- A \wedge B \to B \wedge A
}
\end{tabular}
\end{center}

\end{frame}

%%%%%%%%%%%%%%%%%%%%%%%%%

\begin{frame}
\frametitle{System naturalnej dedukcji}

\begin{center}
\begin{tabular}{lr}
\inference[I$\wedge$]{
\Gamma |- A \qquad \Gamma |- B
}
{
\Gamma |- A \wedge B
}
&
\inference[E$\wedge_1$]{
\Gamma |- A \wedge B
}
{
\Gamma |- A 
}
\end{tabular}
\end{center}

\begin{center}
\begin{tabular}{lr}

\inference[I$\vee_1$]{
\Gamma |- A 
}
{
\Gamma |- A \vee B
}
&
\inference[E$\vee$]{
\Gamma |- A \vee B \qquad \Gamma, A |- C \qquad \Gamma, B |- C
}
{
\Gamma |- C
}

\end{tabular}
\end{center}

\begin{center}
\begin{tabular}{lr}

\inference[I$\to$]{
\Gamma, A |- B
}
{
\Gamma |- A \to B
}
&
\inference[E$\to$]{
\Gamma |- A \to B \qquad \Gamma |- A
}
{
\Gamma |- B
}

\end{tabular}
\end{center}


\begin{center}
\begin{tabular}{c}
\inference[AX]{
}
{
\Gamma, A |- A 
}
\end{tabular}
\end{center}

\end{frame}

%%%%%%%%%%%%%%%%%%%%%%%%%

\begin{frame}
\frametitle{Izomorfizm Curry-Howard (proposition as types)}

\begin{itemize}
 \item Martin-L\"{o}f, Curry, Howard, deBruijn i wiele innych
 \item typy oznaczają formuły
 \item otypowane termy oznaczają dowody dla swoich typów (formuł)
 \item izomorfizm pomiędzy wyprowadzeniami formuł w logice intuicjonistycznej a
       sądami w~systemie typów
 \item realizacja BHK
\end{itemize}

\begin{center}
\begin{tabular}{lr}
\inference[I$\to$]{
\Gamma, x : A |- t : B
}
{
\Gamma |- \lambda x. t : A \to B
}
&
\inference[E$\to$]{
\Gamma |- f : A \to B \qquad \Gamma |- x : A
}
{
\Gamma |- f \; x : B
}
\end{tabular}
\end{center}



\end{frame}

%%%%%%%%%%%%%%%%%%%%%%%%%

\begin{frame}
\frametitle{System typów dla $\lambda$-rachunku}

\begin{center}
\begin{tabular}{lr}
\inference[I$\wedge$]{
\Gamma |- M : A \qquad \Gamma |- N : B
}
{
\Gamma |- (M, N) : A \wedge B
}
&
\inference[E$\wedge_1$]{
\Gamma |- M : A \wedge B
}
{
\Gamma |- fst\;M : A 
}
\end{tabular}
\end{center}

\begin{center}
\begin{tabular}{lr}

\inference[I$\to$]{
\Gamma, x : A |- M : B
}
{
\Gamma |- \lambda x. M : A \to B
}
&
\inference[E$\to$]{
\Gamma |- M : A \to B \qquad \Gamma |- N : A
}
{
\Gamma |- M\; N : B
}

\end{tabular}
\end{center}


\begin{center}
\begin{tabular}{c}

\inference[I$\vee_1$]{
\Gamma |- M : A 
}
{
\Gamma |- inl\; M : A \vee B
}
\end{tabular}
\end{center}

\begin{center}
\begin{tabular}{c}

\inference[E$\vee$]{
\Gamma |- M : A \vee B \qquad \Gamma, x : A |- P : C \qquad \Gamma, x : B |- Q : C
}
{
\Gamma |- when\;M\;(\lambda x.P)\;(\lambda x.Q) : C
}

\end{tabular}
\end{center}


\begin{center}
\begin{tabular}{c}
\inference[AX]{
}
{
\Gamma, x : A |- x : A 
}
\end{tabular}
\end{center}

\end{frame}


%%%%%%%%%%%%%%%%%%%%%%%%%
%%%%%%%%%%%%%%%%%%%%%%%%%
%%%%%%%%%%%%%%%%%%%%%%%%%
%%%%%%%%%%%%%%%%%%%%%%%%%
%%%%%%%%%%%%%%%%%%%%%%%%%
%%%%%%%%%%%%%%%%%%%%%%%%%
%%%%%%%%%%%%%%%%%%%%%%%%%
%%%%%%%%%%%%%%%%%%%%%%%%%
%%%%%%%%%%%%%%%%%%%%%%%%%
%%%%%%%%%%%%%%%%%%%%%%%%%
%%%%%%%%%%%%%%%%%%%%%%%%%
%%%%%%%%%%%%%%%%%%%%%%%%%
%%%%%%%%%%%%%%%%%%%%%%%%%
%%%%%%%%%%%%%%%%%%%%%%%%%


\section{Teoria typów Martina-L\"{o}fa}

\begin{frame}
\frametitle{Teoria typów: historia, idee, początki}

\begin{itemize}
 \item Teoria typów jako logika matematyczna - B. Russel
 \[
  A = \{ w \mid w \not\in w \}
 \]
\pause
 \item $\lambda$-rachunek , funkcja jako pojęcie pierwotne  - A.Church
 \item System typów, likwidacja paradoksu Kleene'go
\[
 K = \lambda x.\; \neg (x\; x)
\]

\[
 (K\; K) = \neg (K\; K) = \neg \neg (K\; K) = \cdots
\]

\end{itemize}


\end{frame}

%%%%%%%%%%%%%%%%%%%%%%%%%

\begin{frame}
\frametitle{Teoria typów Martina-L\"{o}fa}

\begin{itemize}
\item Per Martin-L\"{o}f, A Theory of Types, 1972
\item Per Martin-L\"{o}f, Constructive Mathematics and Computer Programming, 1979
\item Per Martin-L\"{o}f, \textbf{Intuitionistic Type Theory}, 1980
\item Per Martin-L\"{o}f, Truth of a Proposition, Evidence of a Judgement, Validity of a Proof, 1985
\item B. Nordstr\"{o}m, K. Petersson, and Jan M. Smith, \textbf{ Programming in Martin-L\"{o}f's Type Theory: An Introduction}, 1990
\end{itemize}

\end{frame}

\begin{frame}
\frametitle{Wyrażenia jakimi się posługujemy}

\begin{itemize}
\item język wyrażeń:
\begin{itemize}
\item $e(e')$ - aplikacja ( $e(e_1, \cdots, e_n)$ skrócony zapis $e(e_1)\cdots(e_n)$ )
\item $(x)e$ - abstrakcja
\item stałe, np $\lambda$, $\Pi$, $apply, 0, succ$
%\item $(e_1, \cdots, e_n)$ - kombinacja
%\item $(e_1, \cdots, e_n).i$ - selekcja, gdy $1 \le i \le n$
\end{itemize}

\end{itemize}

\end{frame}


\begin{frame}
\frametitle{Wyrażenia jakimi się posługujemy}

\begin{itemize}

\item Pozwalamy na definiowanie nowych stałych jako makra, np
\[
 double \equiv (n)(n + n)
\]
\[
 double(n) \equiv n + n
\]


\item Równość definicyjna
\begin{itemize}
 \item relacja równoważności, kongruencja
 \item surowa równość na wyrażeniach
 \item nic nie mówi o znaczeniu
 \item elementy równe definicyjne traktujemy jako synonimy
\end{itemize}

\begin{eqnarray*}
((x)b)(a) & \equiv & b[x := a]
\\
 ((x)b(x)) &\equiv& b \qquad \mbox{$x$ nie występuje w $b$}
\\
  ((x)b) & \equiv& (y)(b[x := y])
\end{eqnarray*}

\end{itemize}

\end{frame}

%%%%%%%%%%%%%%%%%%%%%%%%%
%%%%%%%%%%%%%%%%%%%%%%%%%
%%%%%%%%%%%%%%%%%%%%%%%%%
%%%%%%%%%%%%%%%%%%%%%%%%%
%%%%%%%%%%%%%%%%%%%%%%%%%
%%%%%%%%%%%%%%%%%%%%%%%%%

\begin{frame}
\frametitle{Wzbogacony system typów o więcej sądów}

\begin{itemize}
 \item Teoria typów Martina-L\"{o}fa - system w~stylu naturalnej dedukcji, w którym możemy wydawać różne sądy:
       
\begin{itemize}
 \item $A$\;set -  jest zbiorem
 \item $a \in A$ - jest elementem zbioru
 \item $a =_A b \in A$ elementy $a$ oraz $b$ są równe w~zbiorze A
 \item $A = B$ - zbiory $A$ oraz $B$ są równymi zbiorami
\end{itemize}

\item elementy zbiorów dzielimy na
\begin{itemize}
 \item kanoniczne - wartości (postać normalna)
 \item niekanoniczne - obliczenia 
\end{itemize}


\item sformułowanie zbioru to
\begin{itemize}
 \item określenie kanonicznych elementów jakie ten zbiór zawiera
 \item określenie co znaczy że dwa elementy są równe w~tym zbiorze
 \item określenie obliczeń
\end{itemize}

\end{itemize}

\end{frame}

%%%%%%%%%%%%%%%%%%%%%%%%%

\begin{frame}
\frametitle{Sądy mają więcej interpretacji}

\begin{itemize}
 \item $A$\;set -  jest zbiorem
\begin{itemize}
 \item $a \in A$ - $a$ jest elementem zbioru
\end{itemize}
 \item $A$\;set -  jest problemem, zagadnieniem, zadaniem
\begin{itemize}
 \item $a \in A$ - $a$ jest rozwiązaniem problemu $A$
\end{itemize}
 \item $A$\;prop - jest formułą logiczną
\begin{itemize}
 \item $a \in A$ - $a$ jest dowodem propozycji $A$
 \item $A$\;true - umiemy zrealizować $A$, istnieje dowód $A$
\end{itemize}
 \item $A\;set$  - jest specyfikacją 
\begin{itemize}
 \item $a \in A$ - $a$ jest programem spełniającym specyfikację $A$
\end{itemize}
\end{itemize}

\end{frame}

%%%%%%%%%%%%%%%%%%%%%%%%%
%%%%%%%%%%%%%%%%%%%%%%%%%
%%%%%%%%%%%%%%%%%%%%%%%%%
%%%%%%%%%%%%%%%%%%%%%%%%%

\begin{frame}
\frametitle{Reguły wnioskowania} 

\begin{itemize}
 \item reguły formułowania
 \item reguły wprowadzania
 \item reguły eliminacji
 \item reguły równościowe
\end{itemize}

\begin{center}
\begin{tabular}{lr}
\inference{
 A\;set \qquad B(x) set\;[x \in A]
}{
 (\Pi x \in A) B(x) \;set
}
&
\inference{
 A\;set\;[\Gamma] \qquad B(x) set\;[x \in A, \Delta]
}{
 (\Pi x \in A) B(x) \;set [\Gamma, \Delta]
}
\end{tabular}
\end{center}

\pause

\begin{center}
\begin{tabular}{lr}
\inference{
 A\;set \qquad B\;set
}{
 A \to B\;set
}
&
\inference{
 b(x) \in B\;[x \in A]
}{
 \lambda b \in A \to B
}
\end{tabular}
\end{center}

\pause

\begin{center}
\begin{tabular}{c}
\inference{
A\;set\;[\Gamma] \qquad B\;set\;[\Delta] \quad
 b(x) \in B\;[\Theta, x \in A]
}{
 \lambda b \in A \to B\; [\Gamma, \Delta, \Theta]
}
\end{tabular}
\end{center}


\end{frame}

%%%%%%%%%%%%%%%%%%%%%%%%%
%%%%%%%%%%%%%%%%%%%%%%%%%
%%%%%%%%%%%%%%%%%%%%%%%%%
%%%%%%%%%%%%%%%%%%%%%%%%%

\begin{frame}
\frametitle{Zbiór liczb naturalnych} 

\begin{center}
\begin{tabular}{lcr}
\inference{
}
{
Nat\; set
}
&
\inference{
}
{
0 \in Nat
}
&
\inference{
n \in Nat
}
{
succ(n) \in Nat
}
\end{tabular}
\end{center}


\begin{center}
\begin{tabular}{lr}
\inference{
n = n' \in Nat
}
{
succ(n) = succ(n') \in Nat
}
\end{tabular}
\end{center}

\pause

\begin{center}
\begin{tabular}{c}
\inference{
a \in N \qquad C \qquad d \in C \\
e(x,y) \in C\;[x \in Nat, y \in C]
}
{
natrec(a, d, e) \in C
}
\end{tabular}
\end{center}


\begin{center}
\begin{tabular}{c}
\inference{
 C\;set \qquad C \\
e(x,y) \in C\;[x \in Nat, y C]
}
{
natrec(0, d, e) = d \in C
}
\end{tabular}
\end{center}

\begin{center}
\begin{tabular}{c}
\inference{
a \in N \qquad C\;set \qquad d \in C \\
e(x,y) \in C\;[x \in Nat, y \in C(x)]
}
{
natrec(succ(a), d, e) = e(a, natrec(a,d,e)) \in C
}
\end{tabular}
\end{center}

\end{frame}

%%%%%%%%%%%%%%%%%%%%%%%%%
%%%%%%%%%%%%%%%%%%%%%%%%%
%%%%%%%%%%%%%%%%%%%%%%%%%
%%%%%%%%%%%%%%%%%%%%%%%%%

\begin{frame}
\frametitle{Zbiór liczb naturalnych} 

\begin{center}
\begin{tabular}{lcr}
\inference{
}
{
Nat\; set
}
&
\inference{
}
{
0 \in Nat
}
&
\inference{
n \in Nat
}
{
succ(n) \in Nat
}
\end{tabular}
\end{center}

\begin{center}
\begin{tabular}{lr}
\inference{
n = n' \in Nat
}
{
succ(n) = succ(n') \in Nat
}
\end{tabular}
\end{center}

\begin{center}
\begin{tabular}{c}
\inference{
a \in N \qquad C(v)\;set\;[v \in Nat] \qquad d \in C(0) \\
e(x,y) \in C(succ(x))\;[x \in Nat, y \in C(x)]
}
{
natrec(a, d, e) \in C(a)
}
\end{tabular}
\end{center}


\begin{center}
\begin{tabular}{c}
\inference{
 C(v)\;set\;[v \in Nat] \qquad C(0) \\
e(x,y) \in C(succ(x))\;[x \in Nat, y \in C(x)]
}
{
natrec(0, d, e) = d \in C(a)
}
\end{tabular}
\end{center}

\begin{center}
\begin{tabular}{c}
\inference{
a \in N \qquad C(v)\;set\;[v \in Nat] \qquad d \in C(0) \\
e(x,y) \in C(succ(x))\;[x \in Nat, y \in C(x)]
}
{
natrec(succ(a), d, e) = e(a, natrec(a,d,e)) \in C(succ(a))
}
\end{tabular}
\end{center}

\end{frame}

%%%%%%%%%%%%%%%%%%%%%%%%%
%%%%%%%%%%%%%%%%%%%%%%%%%
%%%%%%%%%%%%%%%%%%%%%%%%%
%%%%%%%%%%%%%%%%%%%%%%%%%
%%%%%%%%%%%%%%%%%%%%%%%%%

\begin{frame}
\frametitle{Zbiór liczb naturalnych} 

\begin{center}
\begin{tabular}{c}
\inference{
a \in N \qquad C(v)\;set\;[v \in Nat] \qquad d \in C(0) \\
e(x,y) \in C(succ(x))\;[x \in Nat, y \in C(x)]
}
{
natrec(a, d, e) \in C(a)
}
\end{tabular}
\end{center}

\begin{center}
\begin{tabular}{c}
\inference{
a \in N \qquad C(v)\;prop\;[v \in Nat] \qquad C(0)\;true \\
C(succ(x))\;true\;[x \in Nat, C(x)\;true]
}
{
C(a)\;true
}
\end{tabular}
\end{center}

\end{frame}

%%%%%%%%%%%%%%%%%%%%%%%%%
%%%%%%%%%%%%%%%%%%%%%%%%%


%%%%%%%%%%%%%%%%%%%%%%%%%
%%%%%%%%%%%%%%%%%%%%%%%%%


\begin{frame}
\frametitle{Produkt indeksowanej rodziny zbiorów (w matematyce)} 

\begin{itemize}

 \item matematyczna definicja


\[ \prod_{x \in A} B_x = \left\{ f : A \to \bigcup_{x \in A} B_x \mid \forall x \in A.\;f(x) \in B_x\right\}  \]

 \item zależność przeciwdziedziny od argumentu

\[ \mbox{sort} \in \prod_{xs \in Lists} \{ ys \in Lists \mid perm(ys, xs) \wedge sorted(ys) \}  \]

\[ perm(sort(xs_0), xs_0) \wedge sorted(sort(xs_0)) \]

 \item możemy też wyrazić ,,zwykły'' zbiór funkcji, jeżeli $B_x = B$ to

\[ \prod_{x \in A} B_x = A \to B \]

\end{itemize}


\end{frame}

%%%%%%%%%%%%%%%%%%%%%%%%%

\begin{frame}
\frametitle{Produkt indeksowanej rodziny zbiorów}

\begin{center}
\begin{tabular}{lr}
\inference{
A\;set \qquad B(x)\; set\; [x \in A]
}
{
(\Pi x \in A) B(x)\; set
}
&
\inference{
b(x) \in B(x)\;[x \in A]
}
{
\lambda x.b \in (\Pi x \in A) B(x)\; set
}
\end{tabular}
\end{center}

\pause

\begin{center}
\begin{tabular}{c}
\inference{
A = A' \qquad B(x) = B'(x) \;[x \in A]
}
{
(\Pi x \in A) B(x) = (\Pi x \in A') B'(x)
}
\end{tabular}
\end{center}

\begin{center}
\begin{tabular}{c}
\inference{
b(x) = b'(x) \in B(x)\;[x \in A]
}
{
\lambda x.b = \lambda x. b' \in (\Pi x \in A) B(x)\; set
}
\end{tabular}
\end{center}
\end{frame}

%%%%%%%%%%%%%%%%%%%%%%%%%

\begin{frame}
\frametitle{Produkt indeksowanej rodziny zbiorów}


\begin{center}
\begin{tabular}{lr}
\inference{
f \in (\Pi x \in A) B(x)\qquad a \in A
}
{
apply(f,a) \in B(a)
}
&
\inference{
b(x) \in B(x)\;[x \in A] \qquad a \in A
}
{
apply(\lambda x. b , a) = b(a) \in B(a)
}
\end{tabular}
\end{center}

\begin{center}
\begin{tabular}{c}
\inference{
a = a' \in A \qquad f = f' \in (\Pi x \in A) B(x)
}
{
apply(f,a) = apply(f', a') \in B(a)
}
\end{tabular}
\end{center}

\end{frame}

%%%%%%%%%%%%%%%%%%%%%%%%%

\begin{frame}
\frametitle{Produkt indeksowanej rodziny zbiorów}
\[ (\forall x \in A) B(x) \equiv (\Pi x \in A) B(x) \]
\pause
\begin{center}
\begin{tabular}{lcr}
\inference{
A\;set \qquad B(x)\; set\; [x \in A]
}
{
(\Pi x \in A) B(x)\; set
}
&
$\Rightarrow$
&
\inference{
A\;set \qquad B(x)\; prop\; [x \in A]
}
{
(\forall x \in A) B(x)\; prop
}
\end{tabular}
\end{center}

\pause

\begin{center}
\begin{tabular}{lcr}
\inference{
b(x) \in B(x)\;[x \in A]
}
{
\lambda x.b \in (\Pi x \in A) B(x)\; set
}
&
$\Rightarrow$
&
\inference{
B(x)\;true\;[x \in A]
}
{
(\forall x \in A) B(x)\; true
}
\end{tabular}
\end{center}

\pause

\begin{center}
\begin{tabular}{lcr}
\inference{
f \in (\Pi x \in A) B(x)\qquad a \in A
}
{
apply(f,a) \in B(a)
}
&
$\Rightarrow$
&
\inference{
(\forall x \in A) B(x)\;true\qquad a \in A
}
{
B(a)\;true
}
\end{tabular}
\end{center}


\end{frame}

\begin{frame}
\frametitle{Produkt indeksowanej rodziny zbiorów}

\begin{small}
\begin{center}
\begin{tabular}{c}
\inference
{
\inference
{
(\forall x \in A)(\forall y \in B) P(x,y)\;true
}
{
\vdots
}
}
{
(\forall y \in B)(\forall x \in A) P(x,y)\;true
}
\end{tabular}
\end{center}
\end{small}

\pause 

\begin{itemize}
 \item Załóżmy że $f \in (\Pi x \in A) (\Pi y \in B) P(x,y)$, oraz że $x$ nie występuje w $B$
\end{itemize}

\begin{small}
\begin{center}
\begin{tabular}{c}
\inference
{
f \in (\Pi x \in A) (\Pi y \in B) P(x,y) \qquad x \in A\;[x \in A]
}
{
\underbrace{apply(f,x) \in (\Pi y \in B) P(x,y)\; [x \in A]}_D
}
\end{tabular}
\end{center}

\begin{center}
\begin{tabular}{c}
\inference
{
\inference
{
\inference
{
\overbrace{apply(f,x) \in (\Pi y \in B) P(x,y) \; [x \in A]}^D
\qquad y \in B \; [y \in B]
}
{
apply(apply(f,x),y) \in P(x,y) \; [y \in B, x \in A]
}
}
{
\lambda x.\; apply(apply(f,x),y) \in (\Pi x \in B) P(x,y) \; [y \in B]
}
}
{
\lambda y. \lambda x.\; apply(apply(f,x),y) \in (\Pi y \in B)(\Pi x \in A) P(x,y)
}

\end{tabular}
\end{center}



\end{small}

\end{frame}



%%%%%%%%%%%%%%%%%%%%%%%%%


\begin{frame}
\frametitle{Produkt indeksowanej rodziny zbiorów}

\begin{itemize}
 \item Jeżeli $x$ nie występuje w $B$ to
\[ A \to B \equiv (\Pi x \in A) B  \] 
\end{itemize}

\pause

\begin{center}
\begin{tabular}{lr}
\inference{
A\;set \qquad B\; set\;[x \in A]
}
{
A \to B\; set
}
&
\inference{
b(x) \in B\;[x \in A]
}
{
\lambda x.b \in A \to B
}
\end{tabular}
\end{center}

\pause

\begin{center}
\begin{tabular}{lr}
\inference{
A\;prop \qquad B\; prop\;[A\;true]
}
{
A \to B\; prop
}
&
\inference{
B\;true\;[A\; true]
}
{
A \to B\;true
}
\end{tabular}
\end{center}

\end{frame}


%%%%%%%%%%%%%%%%%%%%%%%%%
%%%%%%%%%%%%%%%%%%%%%%%%%
%%%%%%%%%%%%%%%%%%%%%%%%%

\begin{frame}
\frametitle{Suma rozłączna indeksowanej rodziny zbiorów}

\[
 \sum_{x \in A} B_x = \left\{ (a,b) \mid a \in A \wedge b \in B_a \right\}
\]

\pause

\begin{center}
\begin{tabular}{c}
\inference{
A\;set \qquad B(x)\; set\; [x \in A]
}
{
(\Sigma x \in A) B(x)\; set
}
\end{tabular}
\end{center}

\begin{center}
\begin{tabular}{c}
\inference{
a \in A
\qquad
p \in B(a)
}
{
\langle a,p \rangle \in (\Sigma x \in A) B(x)
}
\end{tabular}
\end{center}

\pause


\begin{center}
\begin{tabular}{c}
\inference{
A = A' \qquad B(x) = B'(x) \;[x \in A]
}
{
(\Sigma x \in A) B(x) = (\Sigma x \in A') B'(x)
}
\end{tabular}
\end{center}

\begin{center}
\begin{tabular}{c}
\inference{
a = a' \in A \qquad b = b' \in B(a)
}
{
\langle a , b \rangle = \langle a', b' \rangle \in (\Sigma x \in A) B(x)\; set
}
\end{tabular}
\end{center}
\end{frame}

%%%%%%%%%%%%%%%%%%%%%%%%%

\begin{frame}
\frametitle{Suma rozłączna indeksowanej rodziny zbiorów}


\begin{center}
\begin{tabular}{c}
\inference{
c \in (\Sigma x \in A) B(x) \qquad
C(v)\;set\;[v \in  (\Sigma x \in A) B(x)] \\
d(x,y) \in C(\langle x, y \rangle) \;[x \in A, y \in B(a)]
}
{
split(c,d) \in C(c)
}
\end{tabular}
\end{center}

\pause

\begin{center}
\begin{tabular}{c}
\inference{
a \in A \qquad b \in B(a) \\
C(v)\;set\;[v \in  (\Sigma x \in A) B(x)] \\
d(x,y) \in C(\langle x, y \rangle) \;[x \in A, y \in B(a)]
}
{
split(\langle a , b \rangle, d) = d(a,b) \in C(\langle a , b \rangle)
}
\end{tabular}
\end{center}

\pause

\begin{center}
\begin{tabular}{c}
\inference{
c  = c'\in (\Sigma x \in A) B(x) \qquad
C(v)\;set\;[v \in  (\Sigma x \in A) B(x)] \\
d(x,y) = d'(x,y)\in C(\langle x, y \rangle) \;[x \in A, y \in B(a)]
}
{
split(c,d) = split(c', d')\in C(c)
}
\end{tabular}
\end{center}

\end{frame}

%%%%%%%%%%%%%%%%%%%%%%%%%

\begin{frame}
\frametitle{Suma rozłączna indeksowanej rodziny zbiorów}

\[
 (\exists x \in A) B(x) \equiv (\Sigma x \in A) B(x)
\]

\pause 
\begin{center}
\begin{tabular}{lcr}
\inference{
A\;set \qquad B(x)\; set\; [x \in A]
}
{
(\Sigma x \in A) B(x)\; set
}
&
$\Rightarrow$
&
\inference{
A\;set \qquad B(x)\; prop\; [x \in A]
}
{
(\exists x \in A) B(x)\; prop
}
\end{tabular}
\end{center}

\pause

\begin{center}
\begin{tabular}{lcr}
\inference{
a \in A \qquad p \in B(a) 
}
{
\langle a,p \rangle \in (\Sigma x \in A) B(x)
}
&
$\Rightarrow$
&
\inference{
a \in A
\qquad
B(a)\; true
}
{
(\exists x \in A) B(x)\;true
}
\end{tabular}
\end{center}

\pause

\begin{center}
\begin{tabular}{c}
\inference{
c \in (\Sigma x \in A) B(x) \qquad
C(v)\;set\;[v \in  (\Sigma x \in A) B(x)] \\
d(x,y) \in C(\langle x, y \rangle) \;[x \in A, y \in B(a)]
}
{
split(c,d) \in C(c)
}
\end{tabular}
\end{center}

\begin{center}
\begin{tabular}{c}
\inference{
\Downarrow \\
(\exists x \in A) B(x)\;true \qquad
C\;prop \qquad
C\;true \;[x \in A, B(a) \; true]
}
{
C\;true
}
\end{tabular}
\end{center}


\end{frame}


%%%%%%%%%%%%%%%%%%%%%%%%%

\begin{frame}
\frametitle{Suma rozłączna indeksowanej rodziny zbiorów}
\begin{itemize}
 \item Jeżeli $x$ nie występuje w $B$ to
\[
 A \times B \equiv A \wedge B \equiv (\Sigma x \in A) B
\]
\end{itemize}

\begin{center}
\begin{tabular}{lr}

\inference{
A\;set \qquad B\; set\;[x \in A]
}
{
A \times B\; set
}
&
\inference{
a \in A \qquad b \in B 
}
{
\langle a, b \rangle \in A \times B
}
\end{tabular}
\end{center}

\begin{center}
\begin{tabular}{lr}
\inference{
A\;prop \qquad B\; prop\;[A\;true]
}
{
A \wedge B\; prop
}
&
\inference{
A\;true\qquad B\; true
}
{
A \wedge B\;true
}
\end{tabular}
\end{center}

\end{frame}


%%%%%%%%%%%%%%%%%%%%%%%%%
%%%%%%%%%%%%%%%%%%%%%%%%%

\begin{frame}
\frametitle{Suma rozłączna indeksowanej rodziny zbiorów}

\begin{itemize}
 \item $fst(p) \equiv split(p, (x,y) x )$
 \item Niech $A$,$B$ będą poprawnymi typami, niech $\langle a , b \rangle \in A \times B$
 \item Niech $d(x,y) = x$
 \item NIech $C(p) = A$
\end{itemize}


\begin{center}
\begin{tabular}{c}
\inference
{
a \in A \qquad b \in B \qquad C(v)\;set\; [v \in A \times B] \\
d(x,y) \in C(\langle x , y \rangle) \; [x \in A, y \in B]
}
{
split(\langle a , b\rangle , (x,y)d(x,y) ) = d(a,b) \in C(\langle a , b\rangle)
}
\end{tabular}
\end{center}

\begin{center}
\begin{tabular}{c}
\inference
{
a \in A \qquad b \in B \qquad A\;set\; [v \in A \times B] \\
x \in A \; [x \in A, y \in B]
}
{
split(\langle a , b\rangle , (x,y)d(x,y) ) = a \in A
}
\end{tabular}
\end{center}


\end{frame}


\end{document}
