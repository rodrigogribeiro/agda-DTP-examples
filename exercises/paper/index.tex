\documentclass[11pt, a4paper]{article}
\usepackage{a4wide}
\usepackage{polski}
\usepackage[utf8]{inputenc}
\usepackage{amsthm}
\usepackage{amssymb}

\usepackage{semantic}

\newtheorem{zadanie}{Zadanie}

\author{Wojciech Jedynak \and Paweł Wieczorek}
\title{Ćwiczenia}

\begin{document}

\maketitle

\begin{zadanie}
Dodaj nowe zbiory do naszego systemu - listę parametryzowaną zbiorem $A$ oraz drzewo binarne parametryzowane
zbiorem $A$, tzn sformuuj wszystkie potrzebne reguły aby takie zbiory wprowadzić.
\end{zadanie}

\begin{zadanie}
Na seminarium nie wprowadziliśmy zbioru odpowiadającego spójnikowi $\vee$, pokaż jak dodać taki zbiór.
Zadbaj aby jego reguły odpowiadały regułom z~systemu naturalnej dedukcji.
\end{zadanie}

\begin{zadanie}
Ustalmy typy $A$ oraz rodzinę $B(a)$ dla każdego $a \in A$ oraz $C(a, a')$ dla każdych $a, a' \in A$.
Zbuduj termy o następujących typach:

\[
 \big( (\Pi x \in A)\;(\Pi y \in A)\; C(x, y) \big) \to (\Pi y \in A)\;(\Pi x \in A)\; C(x, y)\;
\]
\[
 \big( (\Sigma x \in A)\;(\Pi y \in A)\; C(x, y) \big) \to (\Pi y \in A)\;(\Sigma x \in A)\; C(x, y)\;
\]
\[
 \big( (\Pi x \in A)\;(\Pi y \in A)\;B(x) \to B(y)\big) \to (\Pi x \in A)\;(\Pi y \in A)\;\neg B(y) \to \neg B(x)
\]
\[
 \big( \neg (\Sigma x \in A)\;B(x)\big) \to (\Pi x \in A)\;\neg B(x)
\]
\end{zadanie}


\begin{zadanie}
Stwórz wyrażenie $compose$, za pomocą którego stworzymy następującą regułę:

\begin{center}
\begin{tabular}{c}
\inference{
g \in A \to B \qquad f \in B \to C
}
{
compose(f, g) \in A \to C
}
\end{tabular}
\end{center}

Następnie zaproponuj regułę dla wersji z typami zależnymi i zdefiniuj odpowiednie wyrażenie $composeDep$.

\begin{center}
\begin{tabular}{c}
\inference{
g \in (\Pi x \in A) B(x) \qquad f \in (\Pi x \in A)(\Pi b \in B(x))\; C(x, b)
}
{
composeDep(f,g) \in\; ?
}
\end{tabular}
\end{center}

Zdefiniuj też pomocnicze wyrażenie $apply2$:

\begin{center}
\begin{tabular}{c}
\inference{
f \in (\Pi x \in A)(\Pi y \in B(x))C(x,y)\qquad x \in A\qquad y \in B(x)
}
{
apply2(f,x,y) \in\; C(x,y)
}
\end{tabular}
\end{center}



\end{zadanie}

\begin{zadanie}
Ustalmy typ $A$, wyprowadź termy o następujących typach.

\[
 (\Pi x \in A)\;[x =_A x]
\]
\[
 (\Pi b \in Bool)(\Pi c \in A)\;[if\;b\;then\;c\;else\;c =_A c]
\]

\end{zadanie}

\begin{zadanie}
Wyprowadź samodzielnie regułę dla prawa Leibniza ($subst$). Następnie posługując się tą regułą pokaż
jak stworzyć reguły dla symetrii i przechodniości. 

\begin{center}
\begin{tabular}{c}
\inference{
P(x)\;set\;[x \in A] \qquad a \in A \qquad b \in A \qquad c \in [a =_A b] \qquad p \in P(a)
}
{
 subst(c,p) \in P(b)
}
\end{tabular}
\end{center}

Oraz pokaż jak stworzyć $cong$ dla poniższej reguły.

\begin{center}
\begin{tabular}{c}
\inference{
f \in A \to B \qquad a \in A \qquad b \in A \qquad c \in [a =_A b]
}
{
 cong(c,f) \in [apply(f,a) =_B apply(f,b)]	
}
\end{tabular}
\end{center}


\end{zadanie}

\begin{zadanie}
 Zdefiniuj funkcję dodawania $add \in Nat \to Nat \to Nat$, a następnie stwórz termy o następujących typach:
\[
 (\Pi a \in Nat)\; [apply2(add, 0, a) =_{Nat}  a]
\]
\[
 (\Pi a \in Nat)\; [apply2(add, a, 0) =_{Nat}  a]
\]
\[
 (\Pi a \in Nat) (\Pi b \in Nat)\; [apply2(add, succ(a), b) =_{Nat}  succ(apply2(add, a, b))]
\]
\[
 (\Pi a \in Nat) (\Pi b \in Nat)\; [apply2(add, a, b) =_{Nat}  apply2(add, b, a)]
\]

Powyższe równości mogą być traktowane jako specyfikacja dodawania, można teraz zauważyć że nasz system z~typami
zależnymi posłużył jednocześnie do zdefiniowania funkcji, zdefiniowania specyfikacji (za pomocą typów identycznościowych)
oraz udowodnił że funkcja spełnia te równości -- co oznacza że wszystko zostało zweryfikowane w obrębie jednego systemu.
\end{zadanie}

\begin{zadanie}
Niech $f \in (\Pi x \in A) B(x)$. Wyprowadź $[f =_{(\Pi x \in A) B(x)} \lambda x. apply(f,x)]$\footnote{
skrócony zapis: $\lambda x. e \equiv \lambda ((x)e)$}.
Następnie zapoznaj się z typem $Eq$ zdefiniowanym w książce (to typ identycznościowy z silną regułą
eliminacji), następnie używając tego typu wyprowadź sąd $f = \lambda x. apply(f,x) \in {(\Pi x \in A) B(x)}$.

Potrzebna jest dodatkowa reguła eliminacji dla typu funkcyjnego, jest to reguła eliminacji
podobna do reguł dla typów indukcyjnych. Mamy rodzinę zbiorów $C$ poindeksowaną funkcjami.
\begin{center}
\begin{tabular}{lr}
\inference{
 f \in (\Pi x \in A) B(x) \\
 C(v)\;set\;[v \in (\Pi x \in A) B(x)] \\
 d(y) \in C(\lambda y)\;[ y(x) \in B(x) \;[x \in A]]
}
{
 funsplit(f,d) \in C(f)
}
&
\inference{
 b(x) \in B(x)\;[x \in A] \\
 C(v)\;set\;[v \in (\Pi x \in A) B(x)] \\
 d(y) \in C(\lambda y)\;[ y(x) \in B(x) \;[x \in A]]
}
{
 funsplit(\lambda b,d) = d(b)\in C(\lambda b)
}
\end{tabular}
\end{center}
Spróbuj zdefiniować $apply$ za pomocą $funsplit$.


\end{zadanie}

\begin{zadanie}
 Udowodnij następujące twierdzenia:

\[
 \Big( (\Pi x \in A) Eq(B(x),\; apply(f,x),\; apply(g,x)) \Big) \to Eq( (\Pi x \in A)B(x),\; f,\; g)
\]
\[
 Eq( (\Pi x \in A)B(x),\; f,\; g) \to \Big( (\Pi x \in A) Eq(B(x),\; apply(f,x),\; apply(g,x)) \Big)
\]
\[
 [f =_{(\Pi x \in A)B(x)} g] \to \Big( (\Pi x \in A) [apply(f,x) =_{B(x)} apply(g,x)] \Big)
\]


\end{zadanie}


\begin{zadanie}
Mogąc robić wyrażenia obliczające typy zobaczyliśmy jak można definiować predykaty, przykładowo: zdefiniowaliśmy
predykat $IsZero(n) = Set(natrec(n, \widehat{\top}, (x,y)\widehat{\bot}))$ który był prawdziwy
tylko dla $0$. Osiągnęliśmy to zwracając $\top$ dla $n=0$ i $\bot$ dla
innych wartości. Spróbuj powtórzyć dowód że $0$ jest różne od $succ(n)$.
\end{zadanie}

\begin{zadanie}
Zrób predykat $Empty$ dla list z pierwszego zadania. Następnie, mając ustalony $A$, zdefiniuj wyrażenie $isEmpty$:

\[
 isEmpty \in (\Pi xs \in List\;A). Empty(xs) \vee \neg Empty(xs)
\]
Zastanów się nad różnicą pomiędzy taką funkcją a poniższą, przypomnij sobie interpretację BHK dla alternatywy.
\[
 isEmpty' \in List\;A \to Bool
\]

Następnie, korzystając z $Empty$, zaprogramuj bezpieczne wersje procedur $head$ oraz $tail$.
\end{zadanie}


\begin{zadanie}
 Skonstruuj negację bitową, tzn funkcję $\mbox{negb} \in Bool \to Bool$ a następnie skontruuj term o typie:
\[
 (\Pi b \in Bool) \neg [b =_{Bool} apply(\mbox{negb},b)]
\]

Dodatkowo, zbuduj termy o następujących typach:

\[
 [apply(negb, true) =_{Bool} false]
\]
\[
 [apply(negb, false) =_{Bool} true]
\]

\end{zadanie}


\begin{zadanie}
 Udowodnij, że nie istnieje funkcja z liczb naturalnych w ciągi zero-jedynkowe, która ma funkcję odwrotną.
To jest skonstruuj
 term o następującym typie:
\[
 \neg (\Sigma f \in N \to BinSeq)\, (\Sigma g \in BinSeq \to N)\,
(\Pi s \in BinSeq)\,[ s =_{BinSeq} apply(f, apply(g, s)) ]
\]
gdzie $BinSeq$ oznacza $N \to Bool$.

Wskazówka: Dowód tego twierdzenia to standardowy przykład metody przekątniowej, można znaleźć rozwiązanie w Whitebooku.
Trudność zadania polega na przeniesieniu rozwiązania do naszego systemu.
\end{zadanie}



\end{document}


